\documentclass[a4]{article}
%
\usepackage{graphics}
\pagestyle{myheadings}
\markright{Copyright @1994-1999 Gwyndaf Evans}
\topmargin=0.0cm
\textheight=20cm
\oddsidemargin=1.0cm
\evensidemargin=1.0cm
\textwidth=14cm
\bibliographystyle{unsrt}
\setcounter{secnumdepth}{1}
%\setcounter{section}{1}
%
%
\begin{document}
\title{{\bf \sc Chooch} \\
Calculation of Anomalous Scattering Factors \\
from X-ray fluorescence data}
\author{Gwyndaf Evans}
\maketitle

\section{Introduction}

The effects of anomalous scattering are described mathematically by
two correction terms which are applied to the normal atomic form
factor or Thompson scattering factor $f_{o}$. The modified scattering
factor is given by $f = f_{o} + f^{\prime} + if^{\prime\prime}$ where
$f^{\prime}$ is the real part and $f^{\prime\prime}$ the imaginary
part of the anomalous scattering correction term.

These anomalous scattering factors vary most rapidly near
characteristic absorption edges of atoms where the energy of the
incident X-rays is similar to the binding energy of the absorbing
electrons. Thought of in classically anomalous scattering is analogous
to any resonance effect such as an electrical LC circuit. 

The optical theorem~\cite{James} says that the imaginary term
$f^{\prime\prime}$ is directly related to the atomic absorption
coefficient for an atom by

\begin{equation}
f^{\prime\prime} = mc \epsilon_{o}E\mu_{a}/e^{2}\hbar
\end{equation}

where $\mu_{a}$ is the atomic absorption coefficient, $E$ the X-ray
energy and all other symbols take there usual meaning. As in other
resonance phenomena such as dielectric susceptibility, the real part
of the dispersive term is related to the imaginary part by a
Kramers-Kronig (K-K) transformation. In the case of X-ray scattering
the K-K transform takes the following form

\begin{equation}
f^{\prime}(Eo)= \frac{2}{\pi}\oint_{o}^{\infty}
                \frac{(E f^{\prime\prime}(E))}{(E_{o}^{2} - E^{2})}dE 
\label{KK}
\end{equation}

\section{Purpose.}

Why do we need to know $f^{\prime\prime}$ and $f^{\prime}$?  When
performing Multiple wavelength Anomalous Diffraction (MAD) experiments
a crucial prerequisite is knowing at which wavelengths to measure
diffraction data. This can only be determined at the time of the
experiment due to two main reasons
\begin{enumerate}
\item For a particular heavy atom element the X-ray
energies to be measured are largely dependent on the environment of
that element within the protein sample and its orientation with
respect to the polarization vector of the incident X-ray beam.
\item The calibration of the incident X-ray energy at different X-ray beam lines
will rarely be the same and as yet no calibration standards have been
established which are common to all crystallographic facilities. 
\end{enumerate}
In addition the calibration of each beam line may vary over time.  As
previously stated the $f^{\prime\prime}$ value is directly related to
the atomic absorption coefficient for an atom. For a discussion of the
difficulties and solutions associated with X-ray energy calibration
for MAD see~\cite{Evans96}. 

The absorption is directly proportional to the X-ray fluorescence
emitted from the atom as a result of absorption of the incident
X-rays. This provides the experimenter with a means of determining the
dependence of $f^{\prime\prime}$ on the X-ray energy. $f^{\prime}$ may
then be determined computationally using the K-K relationship. This
provides the necessary information with which to make a rational
choice of which wavelengths to measure for the experiment. Clearly we
also establish the magnitudes of the anomalous scattering factors as a
function of X-ray energy. These values are potentially useful as
starting points for heavy atom refinement during the latter stages of
data analysis.

\section{Determination of $f^{\prime\prime}$ and $f^{\prime}$}
\subsection{Obtaining $f^{\prime\prime}$ from fluorescence data}

Fluorescence spectra are generally measured directly from the same
frozen protein crystal sample from which the diffraction data is to be
measured.  The spectra are typically recorded using a photo-multiplier
(e.g. Bicron tube) or an energy resolving photo-diode type detector
(e.g. Amptek). In both cases the fluorescence signal is recorded on an
arbitrary scale.  Determination of the corresponding
$f^{\prime\prime}$ spectra is done via two stages. 

Firstly the raw fluorescence spectrum must be background subtracted
and corrected to subtract out any additional scattering effects which
may be energy dependent. This procedure is typically very straight
forward for data measured using a good energy resolving detector such
as the Amptek since the measured signal is essentially dominated by
fluorescent X-ray counts.  However photo-multiplier tubes which have
poor energy resolution will typically measure the elastic scattering
components of the X-rays as well as the fluorescence signal and will
therefore usually require a more careful background correction. 

The procedure involves applying a low order polynomial fit separately
to the below edge region of the spectrum and the above edge region of
the spectrum. The fits should be generated away from the absorption
edge where the XANES effects are smallest. These two polynomials can
then be applied to the raw spectrum such that it is normalized to be
zero fare below the absorption edge and unity above the edge. The
normalized signal $N(E)$ is obtained by
\begin{equation}
N(E) = R(E) \left\{ f^{\prime\prime}_{above}(E)-f^{\prime\prime}_{below}(E) \right \} 
              + f^{\prime\prime}_{below}(E)
\end{equation}
 where R(E) is the raw data, $f^{\prime\prime}_{below}$ is the
polynomial fit in the below edge region and $f^{\prime\prime}_{above}$
the fit for the above edge region. All are functions of the X- ray
energy E.  Theoretical values of $f^{\prime\prime}$ have been
calculated by Cromer \& Libermann~\cite{CromLiberA}. The calculations
however take no account of the effects of coordination of anomalous
scattering atoms to other atoms. The effects of coordination are most
visible in the near edge region which also happens to be the region of
interest for MAD. Therefore the Cromer \& Libermann tables are not
applicable in the near edge region. However away from the absorption
edge above and below in energy the tables provide a good estimate of
the true $f^{\prime\prime}$ values and therefore provide a means by
which the normalized fluorescence data can be converted to a
$f^{\prime\prime}$ spectrum.  The theoretical spectrum is essentially
multiplied into the experimentally determined spectrum to produce an
experimentally determined $f^{\prime\prime}$ spectrum.  

\subsection{Obtaining $f^{\prime}$ from $f^{\prime\prime}$}

Given a $f^{\prime\prime}$
spectrum the K-K transformation may be used to directly obtain a
$f^{\prime}$ spectrum. An algorithm has been
described~\cite{HoytFontaine} which allows this to be carried out
computationally.  Complications arise in the calculation because of
the singularity in the integrand of Equation~\ref{KK} arising when E is equal to Eo and
also because of the impractical limits of integration.  The
singularity is dealt with conveniently by the above algorithm and the
integration limits are chosen such that the calculation remains
possible but does not become inaccurate. Integration limits which
extend only a few keV above and below the absorption edge will usually
provide an accurate estimate of the X-ray energy corresponding to the
minimum value of $f^{\prime}$ but the magnitude of the $f^{\prime}$
curve will in general be incorrect. To obtain highly accurate
magnitudes integration limits are chosen which extend up to $50
\times$ absorption edge energy and to very low energies of say 1
keV. These calculations however are time consuming and not totally
necessary given the experimental requirements. Therefore modest
integration limits may be chosen such that the duration of the
calculation is tolerable as well as the accuracy of the $f^{\prime}$
curve. In the case of the Se K edge recommended integration limits for
a full calculation are 1.2 keV and 630 keV. This calculation takes 120
secs. on a 150 MHz Pentium MMX for a 101 point spectrum. Using
integration limits of $1.2$ keV and $30$ keV takes 8 secs. on the same
CPU and introduces only a $+0.3$ e error into the resulting
$f^{\prime}$ curve.  Such errors are acceptable for the majority of
cases.

\section{Organization of the program}

Calculating anomalous scattering factors from raw fluorescence data
with {\sc Chooch} requires the use of two programs, {\sc Benny} and
{\sc Chooch}. The programs are both called from a shell script, {\tt
Chooch.sh}, which is excecuted by the user. The input and output files
used and generated by the programs are described in detail below.

\subsection{\sc Benny}

{\sc Benny} reads in raw fluorescence data from a measurement perfomed
on a MAD crystal sample and performs a number of manipulative
tasks. Firstly it performs background correction by fitting a
polynomial~\cite{Shampine74} of degree 0 to 3 to the below edge and
above edge regions of the spectrum and normalizing such that the
fluorescence is zero far below the edge and unity far above the edge.
The background fitting step is far from being automatic. The reason
for this is that fluorescence spetra are measured in many different
ways, using different detectors and over differeing energy ranges. This can
give the spectra unusual background properties and make the detection of the
true background level difficult.

The ideal fluorecence spectrum is one where 
\begin{enumerate}
\item the background scatter is low and varies slowly and smoothly
with energy.
\item the data is measured from well below the absorption edge ( $<
(E_{edge}-200)$~eV), to allow the background level below the edge to
be easily established, to well above the edge ( $> E_{edge}+200$~eV),
to establish the level of signal + background above the edge and allow
good normalisation to be performed.
\end{enumerate}

The use of a good energy discriminating detector will often help
satisfy the first criteria. However only the user and beamline staff
can satisfy the second suggestion.

If the spectra has been measured well there will be enough data either
side of the absorption edge to allow a good fit of the backround
levels to be made. The program requires the user to input values of
the X-ray energy between which the background fits will be made. This
is done graphically with the cursor. First click on the low energy
point and the on the high energy point. The user can also choose the
type of fit to be generated by choosing the polynomial order. This
will be either a straight line, a quadratic or a cubic. Choosing the
[0] option requires that the user select the actual background level
with zero slope (a completely manual option reserved for poorly
measured data where it is not possible to determine the background by
fitting).

After normalisation the program goes away and applies a spline
fit~\cite{Woltring86} to the normalized data creating a smooth
curve. It additionally calculates 1st, 2nd and 3rd derivatives of
smoothed data for input into {\sc Chooch}. This step is fully
automatic.  Parameters varying the type of fit performed are hard
coded into the program and have so far served well on all cases that I
have tested the program on so if it should fail please let me know.

\subsubsection{Files}

\begin{description}
%\labelsep=0.5cm
%\itemindent=-0.0cm
%\listparindent=0.0cm
%\leftmargin=4.0cm
\item[{{\it file}.raw}] The raw input fluorescence data to {\sc
Benny}. The first line should contain the number of data points
(integer). The second line in usually blank but is interpreted as
text. It can be used as a comment line for the data.  Each subsequent
line should contain three values referring to one data point - The
data point number (integer), the X-ray energy {\bf in electron-Volts (eV) ( NOT keV)} 
and the fluorescence signal on an arbitrary scale (real).

e.g.
\begin{verbatim}
Fluor. spectrum for element Qu  ; Title (a80)
801                             ; No. data points (free format)
12300.0  2002                   ; Energy (eV), Flu. Signal (free format)
12300.5  2030
12301.0  2035
12301.5  2450
.
.
.
12700.0  6700
\end{verbatim}

\item[splinor] Output by {\sc Benny} and input to {\sc Chooch}
containing the X-ray energy, smoothed normalized fluorescence data
1st, 2nd and 3rd derivatives (format(5f13.3)).

e.g.

\begin{verbatim} 

Fluor. spectrum for element Qu  ; Title (a80)
     101                        ; No. of data points (i8)
    12654.043        0.013       -0.005        0.052       -0.083
    12654.438        0.015        0.009        0.019       -0.081
    12654.831        0.019        0.011       -0.008       -0.051
    12655.226        0.022        0.006       -0.015        0.018
.
.
.
    12693.076        1.004        0.007        0.031        0.047
    12693.471        1.009        0.023        0.050        0.047
\end{verbatim}
\end{description}

\subsection{\sc Chooch}

Takes output from Benny and reads input data about the element and
absorption edge in question from a command file.  It then calculates
$f^{\prime\prime}$ and $f^{\prime}$ from the smoothed normalized
fluorescence data and displays the resulting curve. The program
automatically selects the peak $f^{\prime\prime}$ energy and the
minimum $f^{\prime}$ energy and outputs them.  An important
requirement of the program is that the input spectrum be on a strictly
increasing X-ray energy scale of constant energy increments this
requirement is satisfied however by the smoothing procedure performed
by {\sc Benny}.

In order to extrapolate the $f^{\prime\prime}$ spectrum to very low
and very high energies prior to integration {\sc Chooch} uses the
subroutine {\tt mucal.f} written by Pathikrit Bandyopadhyay to obtain
values of total crossection as published by
McMasters~\cite{McMasters69}. The subroutine may be found at
http://ixs.csrri.iit.edu/database/programs/mcmaster.html.

\subsubsection{Files}

\begin{description}
\item[splinor] Output from {\sc Benny} as described above.

\item[atom.lib] This file contains a compilation of required values
for all atoms and absorption edge of interest to protein
crystallographers. The contents fro each element is demonstrated here
by looking at the entry for selenium.

\begin{verbatim}
SE  -0.215
K    1265.80  50632.00
  12618.00   0.5021  12638.00   0.5006  12678.00   3.8315  12698.00   3.8186
L1    165.30   6612.00
   1613.00  13.7415   1633.00  13.4776   1673.00  14.8676   1693.00  14.6049
L2    147.70   5908.00
   1437.00  37.5640   1457.00  24.5190   1497.00  23.4875   1517.00  18.1565
L3    143.60   5744.00
   1396.00   2.4959   1416.00   2.4452   1456.00  25.0133   1476.00  17.2386
M      23.20    928.00
    192.00  14.5096    212.00  14.9681    252.00  15.0664    272.00  14.5137
\end{verbatim}

The first line contains the atomic symbol of the element followed by
an $f^{\prime}$ correction term $5E_{tot}/3mc^{2}$ as published by
Cromer \& Liberman~\cite{CromLiberA}. The following five pairs of
lines contain information pertaining to each of five absoption edges
(M correcsponds to the $M_{V}$ edge). The first line of each pair
contains the absorption edge name followed by the lower and upper
integration limits needed by {\sc CHOOCH}. The second line has four
pairs of values taken from the CROSSEC program. Two at enegies just
below the edge and two just above the edge. These values are used by
{\sc CHOOCH} to perform the renormalisation of the raw spectrum to the
theoretical values away from the absorption edge. N.B. If for some
reason an element that you require is missing from this file please
contact the author.

\item[{\it file}.efs] Output from {\sc Chooch} containing calculated
anomalous scattering factors. The example below is taken from the
examples directory in the {\sc Chooch} distribution.

e.g.

\begin{verbatim}
Se test data from a foil Chooch test data ; Title (a80)
     101                                  ; No. of data points (i8)
  12654.04      0.51     -6.47            ; Energy, f'', f' (3f10.2)
  12654.44      0.51     -6.52
  12654.83      0.53     -6.58
.
.
.
  12692.29      3.77     -4.81
  12692.68      3.76     -4.79
  12693.08      3.76     -4.72
  12693.47      3.78     -4.69
\end{verbatim}

\item[{\it file}.inf] Output from {\sc Chooch} containing summary of
calculation. 

e.g.

\begin{verbatim}
Se test data from a foil Chooch test data                                       
 Total points integrated     :      71890
 Integration limits low/high :    1653.06  30000.25
 First/last data points at   :   12654.04  12693.47
 Energy scale increment      :  0.394

 Inflection point at   12665.48 with  f` of  -9.7
             Peak at   12667.45 with f`` of   6.6
\end{verbatim}

\item[{\it file}.ps] Output from {\sc Chooch}. Once the anomalous scattering
factors have been calculated and displayed you can select the '{\tt p}' option
and dump a {\sc PostScript} file of the plot. If you do not press '{\tt p}' the file is
still generated unfortunately but will contain inly a pre-amble.

\end{description}


N.B. Out of all the above files only file.raw need be created by the user - all 
other are produced by either {\sc Benny} or {\sc Chooch}.

\section{Installation}

If you are reading this manual the chances are you have already installed the
program and have read the installation guide README.Install

\section{Running the program}

The program should be excecuted in the directory containing the raw
fluorescence data file, {\it file}.{\bf raw} by typing {\tt Chooch.sh}
$<${\bf element}$>$ $<${\bf edge}$>$ {\it file} at the command line prompt. 
\begin{itemize}
\item {\bf element} - two letter atomic symbol (case insensitive)
\item {\bf edge}  - $K | L1 | L2 | L3 | M$
\end{itemize}

For example to run the example fluorescence data named SeFoil.raw type
\begin{verbatim}
Chooch.sh  SE K SeFoil
\end{verbatim}

The c-shell script Chooch.sh prepares a number of small files required by Benny and Chooch
at runtime. If all is properly installed a PGPLOT window will appear displaying
your fluorescence spectrum. You will then be guided through the procedure.
A simple run down of the procedure follows:

\begin{enumerate}
\item Fitting the below edge region - in the PGPLOT window just type
0,1,2 or 3 depending on the type of fit you would like. More often
that not you will use either 0 or 1. Fluorescence data from proteins
is typically measured over a fairly limited range which requires a bit
of guess work when determining the background level (hence the need for
option 0). Anyway, after entering a choice the cursor will appear and if
you chose option 0 click at an appropriate level for the below
edge background. If you chose any other option select two energies - low
first, then high - which ideally should be $E_{edge} - 100$ and
$E_{edge}-25$ or so. If the data doesn't extend that far below then
use option 0 or be very careful!
\item Fitting the above edge region - this is the same as for the
below edge region but you should take care that you ignore any near
edge effects when selecting the energy range for a fit. You shouldn't
bias the fit with a large white line peak. Therefore use caution and
select a low energy which is away from the near edge region where the
XANES ripples begin to die out ($\sim E_{edge}+30$).
\item When inspecting the normalisation result you can decide you don't like it
and by typing '{\tt n}' you can return to the beginning and refit the backgrounds.
\item If you do like it then just click the mouse to continue and the program will smooth
the data for you.
\item Proceed by hitting '{\tt c}' and the PGPLOT window will disappear while {\sc Chooch}
calculates the anomalous scattering factors. When it's complete another PGPLOT window will
appear with the results. It also prints estimates of the $f^{\prime\prime}$ peak energy
and the $f^{\prime}$ minimum energy.
\end{enumerate}


\subsection{Using the zoom facility}

At most stages of the procedure you can zoom in on your spectrum by
pressing '{\tt z}' and selecting a low then a high X-ray energy with the mouse cursor.
The zoomed region will then appear in the same window. You can redraw the original 
data range by pressing '{\tt r}'.

\subsection{Generating a {\sc PostScript} plot of the result}

Once {\sc Chooch} has produced the PGPLOT window containing the
anomalous scattering curves you can generate a plot of the output by selecting
the '{\tt p}' option in the PGPLOT window.


\begin{thebibliography}{1}

\bibitem{James} R.W. James.  \newblock {\em The {O}ptical {P}rinciples
of the {D}iffraction of {X}-rays}.  \newblock G. Bell and sons Ltd,
London, 1969.

\bibitem{Evans96}
G.~Evans and R.~F. Pettifer.
\newblock Stabilisation and calibration of x-ray wavelengths for anomalous
  diffraction experiments using synchrotron radiation.
\newblock {\em Rev. Sci. Instr.}, 67(10), October 1996.

\bibitem{CromLiberA}
D.T. Cromer and D.~Liberman.
\newblock Relativistic calculation of anomalous scattering factors for
  {X}-rays.
\newblock {\em J. Chem. Phys.}, 53:1891--1898, 1970.

\bibitem{HoytFontaine}
J.J. Hoyt, D.~de~Fontaine, and W.K. Warburton.
\newblock Determination of the anomalous scattering factors for {C}u, {N}i and
  {T}i using the dispersion relation.
\newblock {\em J. Appl. Cryst.}, 17:344--351, 1984.

\bibitem{Shampine74}
L.F. Shampine, S.M. Davenport, and R.E. Huddleston.
\newblock Fit discrete data in a least squares sense by polynomials in one
  variable.
\newblock Fortran Subroutine, 1974.

\bibitem{Woltring86}
H.J. Woltring.
\newblock Test programme for generalized cross-validatory spline smoothing with
  subroutine gcvspl and function splder using the data of c.l. vaughan,
  smoothing and differentiation of displacement- time data: an application of
  splines and digital filtering.
\newblock Fortran Subroutine, 1986.

\bibitem{McMasters69}
W.H. McMasters, N.~Kerr, Del~Grande, J.H. Mallett, and J.H. Hubbekk.
\newblock Compilation of {X}-ray cross sections.
\newblock Technical Report UCRL-50174, Lawrence Radiation Laboratory
  (Livermore), 1969.

\end{thebibliography}


\end{document}


